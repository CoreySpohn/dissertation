Observing time for current and future space telescopes is both incredibly
valuable and a finite resource. Gathering the light required for a full
atmospheric detection of a habitable planet around a Sun-like can take weeks of
integration time \citep{TheLUVOIRTeam2019}. The work in this dissertation
investigated the efficient use of observation time for a future mission by
utilizing prior knowledge of target exoplanets. Better usage of observation
time will improve the science yield of future missions that aim to incorporate
direct imaging.

To do that I started in \Cref{cha:first_paper} by developing a method of
creating orbits consistent with radial velocity data that could be used to
schedule observations with a direct imaging instrument. This was done by
creating synthetic radial velocity data and fitting it with modern radial
velocity orbit fitting tools. Then the fitted orbital parameters were used to
create orbits with all the parameters necessary for direct imaging, propagated
in time, and compared to the true detectability of the planet used to generate
the synthetic radial velocity data.

Then in \Cref{cha:coupling} I described a method of calculating the dimmest
planet that can be detected for a specific observing scenario in the case when
there is no analytical solution possible. The problem is formulated as a
numerical minimization problem on the exposure time calculator for a specific
combination of telescope, star, and integration time. This is useful as it can
be used to calculate the probability of a specific planet type being detected
around a specific star. I applied that method to observing scenarios for the
Roman Space Telescope and determined the exoplanets that are most likely to be
detectable. With that method created, in \Cref{cha:accurate_pdet} I improved
the work described in \Cref{cha:first_paper} to calculate the probability of
detection by accounting for different integration times and variations in the
local zodiacal light and exozodiacal light.

The dissertation ends by validating of the probability of detection metric in
\Cref{cha:sim_and_scheduling}. I created a framework for simulating the full
process of collecting and using radial velocity data for direct imaging.
Realistic radial velocity observation runs are performed and blind fitting is
done to find planets from the synthetic radial velocity data. Then the fitted
orbits are used to calculate the probability of detecting the planets in the
simulation. A constraint programming method was used to create an observation
schedule in line with the current goals of a direct imaging mission. By running
full simulations I showed that the method is able to consistently schedule
successful observations of Earth-like exoplanets when they have prior radial
velocity data.

This work proves that it is possible to use the incomplete orbital information
that a radial velocity orbit fit provides to inform a future direct imaging
mission that will search for habitable planets outside of our solar system. The
tools I created also allow for studies on a number of interesting questions
such as observing strategies for radial velocity instruments, systematic
evaluation of different orbit fitting tools, and the impact of reaching
different radial velocity precisions on the number of Earth-like atmospheres a
future direct imaging mission will be able characterize.
