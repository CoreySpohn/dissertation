\section{Motivation}
\label{sec:motivation}

An exoplanet is simply any planet that is not in our solar system. I want to
motivate why we care about them because fundamentally they are very far from
Earth, hard to find, and require very expensive instruments. 

The astronomy community's interest in exoplanets has been growing as the number
of planets detected increases. This interest is easy to demonstrate with the
National Academies of Sciences 2020 decadal report which identified "Worlds and
Suns in Context" as one of the three broad themes that compose "the most
compelling science" in astronomy and astrophysics for the next 10 years
\citep{nationalacademiesofsciencesPathwaysDiscoveryAstronomy2021}. They
identify the discovery of habitable worlds as a priority area and motivate it
as a drive to understand our place in the universe with questions such as: how
rare is life, are there worlds that humans might one day live on, and what can we
learn about Earth by studying other such planets? The answers to these questions
will be profound and they can only be answered through considerable scientific
effort.

The study of non-habitable planets pose fewer existential questions, and are
often overlooked because of it, but they too have an important role to play in
our understanding of the universe. The process of planetary formation has
interested scientists centuries before any planet had been found, in fact both
Immanuel Kant and Pierre-Simon Laplace proposed theories \citep{Perryman2018a}
in the 1700s. Our models today are built around a three stage process where a
protoplanetary disk collapses, planetesimals form through accretion, and the
planetesimals ultimately merge through gravity. This process is generally
accepted but the specifics of the steps, particularly the accretion step, are
not well understood \citep{Perryman2018a}. Every planet detected can be used to
fine-tune our models of that formation process.

There's also a societal purpose to the study of exoplanets, as fostering a
public interest in science and education is beneficial to the well being of
humans broadly. It is difficult to judge the impact of exoplanet science alone
on the public's interest in science, but \citet{deegImpactExoplanet2018} notes
that internet search activity shows a wide interest in exoplanet science,
particularly for habitable worlds.

% \textbf{Why care about exoplanets?}
% - Life outside Earth
% - Understanding the formation and evolution of planets
% - The occurrence rates of planets
% - Public interest in science
% - Potentially habitable worlds
%   - Understanding how rare Earth is as a motivator to keep it healthy

\section{Exoplanet Detection Methods}
\label{sec:detection_methods}

\subsection{Indirect Detection} 

The first unambiguous exoplanet detection occurred in 1995 through radial
velocity observations of the star 51 Peg \citep{mayorJupitermassCompanion1995}.
The radial velocity method works by observing the Doppler shift in a star's
spectral lines and inferring the motion of the star around a planet-star center
of mass. This method is biased towards massive planets on edge on orbits as the
strength of a radial velocity signal is determined by the factor $M\sin{i}$,
where $M$ is the planet's mass and $i$ is the inclination of the planet's
orbit. The transit photometry exoplanet detection technique finds planets by
carefully monitoring the light coming from planets and watching for dips that
occur when a planet passes in front of the star. The radius and orbital period
of an exoplanet can be determined with this technique and it is therefore
biased towards large radius planets that are close to their star. The
astrometry detection method is done through extremely precise measurements of a
star's position in the planet of the sky \citep{Perryman2018a}. With the
precise movement of the star the corresponding motion of the planets can be
inferred.

Collectively, the transit technique has detected the most planets. At the time
of writing the NASA Exoplanet Archive shows 4114 planets detected via transit,
1044 detected via radial velocity, and two detected with astrometry. These
numbers are steadily increasing and missions such as TESS and Gaia will
increase the number of known exoplanets by the thousands \citep{Perryman2018a}
\citep{Huang2018}.

% - Microlensing

\subsection{Direct Detection}

Direct detection of exoplanets is the process of collecting light directly from
the exoplanet. This can be done when the planet is hot enough to emit light on
its own or when light from the star reflects off of the planet. Direct
detection is biased towards large radius planets at wide separations. Because
of this direct imaging complements the other techniques by broadening the range
of orbital periods that we are able to detect. A further benefit of direct
imaging is its ability to obtain detailed spectral information on planets that
are not detectable via the transit technique. Gathering spectral information
is of primary interest because it allows us to study the atmosphere of the
planet and search for signs of water and biosignatures.

% \textbf{Why focus on radial velocity for direct imaging}

\section{Method Outlook}
\label{sec:EPRV_HWO}
% Make Plot of detections on the Exoplanet Archive

This dissertation will focus on the synergy of using radial velocity data to
predict the best times to make observations of exoplanets with a direct imaging
instrument. This combination was chosen due to the difficulty of getting
spectral information on Earth-like exoplanets. The 2020 decadal report
recommended a 6 meter space-based telescoped optimized for directly detecting
and spectrally characterizing habitable exoplanets
\citep{nationalacademiesofsciencesPathwaysDiscoveryAstronomy2021}. This form of
mission was studied as a part of the HabEx and LUVOIR mission concepts
\citep{gaudiHabitableExoplanetObservatory2020,TheLUVOIRTeam2019}. Both the
HabEx and LUVOIR teams recommended considerable effort be put into finding
habitable worlds through other detection methods before the launch of a
space-based direct imaging mission. The main reason for this is simply that it
is easier to detect a planet when you now approximately where it will be.
Additionally, if a star is observed and no Earth-like planet is found then
that star can be removed from the direct imaging mission's list of stars
to observe, or target list.

Earth-like planets around Sun-like stars are difficult to detect for any
method. They have a very low probability of transiting, astrometry detection of
an Earth-like planet can only be done with a space-based mission two orders of
magnitude more sensitive than the Gaia mission, and the radial velocity method
is similarly approximately two orders of magnitude
away\citep{gaudiHabitableExoplanetObservatory2020}. However, the radial
velocity method can feasibly make the required improvements with ground based
observatories and many efforts to reach the required sensitivity are in
progress\citep{Fischer2016a}. The radial velocity precision required to 
detect an Earth-like planet is approximately 10 cm/s and a number of instruments
have recently demonstrated on-sky radial velocity precision under 1 m/s \citep{maroonx2021,
guptaTargetPrioritization2021,Pepe2021}. Additionally, the Extreme Precision
Radial Velocity initiative report from 2021 concluded that there are multiple
plausible system architectures that can detect Earth-like exoplanets under
the right conditions.

% \section{Yield modeling}
% \textbf{Why create a framework for full simulations}

\section{Dissertation Overview}
\label{sec:dis_overview}

In this work I present what I have done to improve our chances of directly
detecting an exoplanet that has been detected with the radial velocity method.
In \Cref{cha:first_paper} I begin by analyzing what orbital information is
missing, how to fill the information, how to use that information to estimate
the probability of a telescope directly detecting that planet, and what can go
wrong when doing so. Then in \Cref{cha:coupling} I describe a method to
accurately estimate what the dimmest planet a direct imaging mission can detect
in the likely scenario where the planet's brightness has a subtle impact on the
amount of observational noise. With that method in \Cref{cha:accurate_pdet} to
expand the probability of detection method to incorporate more factors to
describe more realistic observation scenarios. Then in
\Cref{cha:sim_and_scheduling}, I validate the probability of detection
calculations with a software tool I created that manages the creation of
synthetic planetary systems, the radial velocity observation process on the
synthetic planetary systems, full orbital fitting of the collected radial
velocity data, calculating the probability of detecting the synthetic planets
that were fitted, scheduling observations based on the probability of
detection, and finally simulating direct imaging observations based on the
created schedule to determine its yield.


